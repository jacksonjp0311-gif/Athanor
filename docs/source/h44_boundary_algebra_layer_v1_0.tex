% =============================================================================
% 𓂀 CODEX H₄₄ — BOUNDARY ALGEBRA LAYER
% =============================================================================
% AUTHOR
%   James Paul Jackson
%   X / Twitter: @unifiedenergy11
%
% TITLE
%   "H₄₄ — Boundary Algebra:
%    A Selection Layer for Extremal Invariants at Feasibility Boundaries"
%
% STATUS
%   CANONICAL CODEX LAYER — v1.0 (LOCKED)
%   This document defines H₄₄ as a persistent structural layer within the
%   Codex H-Layer architecture.
% =============================================================================

\documentclass[11pt]{article}
\usepackage[margin=1in]{geometry}
\usepackage{amsmath,amssymb}
\usepackage{hyperref}
\usepackage{booktabs}

\title{H₄₄ — Boundary Algebra:\\
\large A Selection Layer for Extremal Invariants at Feasibility Boundaries}
\author{James Paul Jackson\\ \texttt{@unifiedenergy11}}
\date{2026}

\begin{document}
\maketitle

\begin{abstract}
H₄₄ (Boundary Algebra) is a Codex structural layer describing how extremal
invariants are selected at sharp or asymptotically sharp feasibility boundaries
across constrained systems. The layer asserts that extremal selection is governed
by boundary geometry rather than bulk dynamics, and that governing relations at
such boundaries collapse to low-degree algebraic structure.
\end{abstract}

\section{Layer Definition}
\textbf{H₄₄ — Boundary Algebra Layer} governs the selection of extremal algebraic
invariants at feasibility boundaries in constrained systems.

\section{Operational Principles}
\subsection*{P1 — Boundary Extremality}
Extremal configurations are selected at feasibility boundaries.

\subsection*{P2 — Algebraic Collapse}
At feasibility boundaries, governing conditions collapse to low-degree algebraic
structure.

\subsection*{P3 — Minimal-Degree Selection}
The extremal invariant selected at the boundary has the minimal algebraic degree
compatible with the constraint geometry.

\subsection*{P4 — Commensurability Suppression}
When rational commensurability destabilizes configurations, the selected
invariant maximizes resistance to rational approximation within its degree class.

\section{Quadratic Boundary Selection (Canonical Case)}
When boundary collapse is quadratic and commensurability suppression is maximal,
H₄₄ predicts selection of the golden ratio
\[
\phi = \frac{1+\sqrt{5}}{2}.
\]

\section{Null Predictions}
\begin{itemize}
\item If no algebraic collapse occurs, no preferred invariant should appear.
\item If collapse is linear, quadratic irrationals should not be selected.
\item If collapse is quadratic but commensurability suppression is weak,
      non-extremal quadratics may appear instead of \(\phi\).
\end{itemize}

\section{Conclusion}
H₄₄ formalizes a boundary-first selection principle within the Codex framework:
collapse is triggered by ΔΦ, stabilized by H₇, and resolved algebraically by
Boundary Algebra.

\end{document}
