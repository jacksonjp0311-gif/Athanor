% ─────────────────────────────────────────────────────────────────────────────
% CODEX–777 TRIADIC BOUNDARY EXCESS PRINCIPLE
% Canonical Geometry Extraction Note (v2.0 · Locked)
%
% Author: James Paul Jackson
% Date: February 4, 2026
%
% STATUS
% ------
% GEOMETRY DOCUMENT — DISCRETE CURVATURE THRESHOLD FORMALIZATION (FINAL).
%
% This note embeds the 777 closure signature into classical tessellation and
% discrete curvature theory:
%
% • {3,6} is the unique Euclidean triangular tiling (flat closure).
% • {3,7} is the first hyperbolic triangular tiling (minimal excess).
% • k* = 7 is the minimal curvature insertion beyond feasibility.
% • The equilateral triangle (C3 symmetry) is the minimal triadic carrier.
%
% v2.0 scope:
% • Boundary-selected integer curvature threshold.
% • Coupling between curvature sign and global growth regime.
% • Finite → infinite refinement transition at minimal excess.
% • Infinite self-similar nesting and multidirectional proliferation.
% • Dimensional asymmetry in tetrahedral honeycombs {3,3,n}.
%
% This is the integer analogue of boundary-extremal selection:
% invariants emerge at sharp feasibility boundaries by minimal defect.
%
% No numerological, perceptual, or metaphysical interpretation is made.
% ─────────────────────────────────────────────────────────────────────────────
\documentclass[12pt]{article}
\usepackage{amsmath,amssymb}
\usepackage{geometry}
\usepackage[pdftex]{hyperref}
\geometry{margin=1in}

\newcommand{\keywords}{
regular tilings; hyperbolic geometry; angle excess;
Regge calculus; discrete curvature; boundary thresholds.
}
\newcommand{\msc}{
Primary 52C20; Secondary 51M20, 53C23.
}

\title{\textbf{The Triadic Boundary Excess Principle (777)}}
\author{\textbf{James Paul Jackson}}
\date{February 4, 2026}

\begin{document}
\maketitle
\vspace{-0.6em}
\noindent \textbf{Keywords:} \keywords

\noindent \textbf{MSC (2020):} \msc

\begin{abstract}
The minimal integer beyond flat triangular closure ($k=6$) is $k^\star=7$,
forcing hyperbolic curvature in the regular tiling $\{3,7\}$. Triadically
replicated under the equilateral carrier’s $C_3$ symmetry, this
boundary-selected defect yields $(7,7,7)\equiv 777$ and couples local curvature
sign to a global qualitative shift: bounded refinement at $\varepsilon_v=0$
versus unbounded exponential self-similarity at minimal negative curvature.
\end{abstract}

\section{Feasibility Boundary in the Equilateral Lattice}

Let $\alpha = 60^\circ$ denote the interior angle of an equilateral triangle.
The unique Euclidean triangular tiling satisfies
\[
\{3,6\}, \qquad 6\alpha = 360^\circ.
\]
The next integer gives
\[
7\alpha = 420^\circ,
\]
exceeding full angular closure, enforcing negative curvature, and establishing
$k^\star=7$ as the minimal threshold for the hyperbolic tiling $\{3,7\}$.

\section{Angular Excess and Minimality}

Define the angular excess functional
\[
\delta(k) := k\alpha - 360^\circ.
\]

\textbf{Lemma.}
For $\alpha=60^\circ$, the unique integer satisfying $\delta(k)=0$ is $k=6$.

Define
\[
k^\star := \min\{k\in\mathbb{N} : \delta(k)>0\}.
\]
Then
\[
k^\star=7, \qquad \delta(7)=60^\circ.
\]

\section{Discrete Curvature and Regge Deficit}

In Regge calculus, curvature at a vertex is encoded by the angle deficit
\[
\varepsilon_v = 2\pi - \sum_{i=1}^k \theta_i.
\]
For equilateral tilings,
\[
\varepsilon_v = 360^\circ - k\alpha = -\delta(k).
\]

Thus:
\begin{itemize}
\item $k=6 \Rightarrow \varepsilon_v=0$ (Euclidean flatness),
\item $k=7 \Rightarrow \varepsilon_v<0$ (hyperbolic curvature).
\end{itemize}

\section{Curvature Sign and Global Growth Regime}

Negative curvature induces exponential growth in smooth hyperbolic geometry.
The same phenomenon appears discretely.

For $\varepsilon_v=0$, refinement in $\{3,6\}$ asymptotically collapses toward
finite limits. When $\varepsilon_v<0$, as in $\{3,7\}$, the number of distinct
cells within combinatorial distance $r$ grows exponentially.

Thus the minimal excess $\delta(7)>0$ couples local curvature sign to a global
phase transition: from bounded refinement to unbounded self-similar growth.

\section{Triadic Carrier and Symmetry Replication}

The equilateral triangle, with rotational symmetry group $C_3$, partitions into
three $120^\circ$ sectors. Sector-local excess $\delta(7)$ replicates identically
across these sectors, yielding the triadic signature
\[
(7,7,7)\equiv 777.
\]

\section{Dimensional Extension and Asymmetry}

In three dimensions, equilateral triangular faces assemble into tetrahedral
honeycombs $\{3,3,n\}$. No integer $n$ yields exact Euclidean closure due to the
irrational tetrahedral dihedral angle.

The smallest integer producing a consistent hyperbolic structure is $n=5$,
corresponding to $\{3,3,5\}$.

\section*{Closing Remark}

The symbol $777$ denotes a boundary-selected curvature controller: the minimal
integer defect that simultaneously selects curvature sign, symmetry replication,
and the onset of infinite discrete structure.

\end{document}
