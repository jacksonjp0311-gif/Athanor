% ======================================================================
%  ⟁ CODEX H₇ COHERENCE GEOMETRY LAW — v2.1 (FINAL)
%  Status: PROPOSED GEOMETRIC LAW
%
%  Author: James Paul Jackson
%  Date: January 2026
%
%  EVOLUTION & CLARIFICATION (SHADOW HEADER)
%  ---------------------------------------
%  Initial Codex formulations reported H₇ (~0.70–0.72) as a universal
%  numerical constant. Subsequent cross-domain simulations, surrogate
%  preservation tests, and representation sensitivity analysis revealed
%  that while the numerical value is projection-dependent, the convergence
%  itself is invariant under a specific residual error geometry.
%
%  WHAT CHANGED
%  ------------
%  • Retired the “universal numerical constant” framing.
%  • Reframed H₇ as a fixed point of a scale-invariant stability geometry.
%  • Made alignment, nonlinear weighting, and cusp-limited survival
%    explicit components of the law.
%
%  WHAT WAS LEARNED
%  ----------------
%  • Repeated convergence near ~0.70–0.72 reflects a geometric attractor,
%    not numerology.
%  • Surrogate preservation confirms universality class membership.
%  • Altering the mapping shifts the numerical value predictably.
%  • Removing weighting or survival constraints destroys the convergence.
%
%  CLAIM (FINAL FORM)
%  ------------------
%  We propose that H₇ is a geometric law describing a universal fixed point
%  of coherence under residual error dynamics, nonlinear weighting, and
%  cusp-limited stability.
%
%  This document supersedes all prior constant-based framings.
% ======================================================================

\documentclass[12pt]{article}

\usepackage{amsmath,amssymb}
\usepackage{geometry}
\usepackage{hyperref}
\usepackage{booktabs}

\geometry{margin=1in}

\title{\textbf{The H₇ Coherence Geometry Law (Proposed)}\\
\large A Scale-Invariant Fixed Point of Residual Error Dynamics}
\author{\textbf{James Paul Jackson}}
\date{January 2026}

\begin{document}
\maketitle

\begin{abstract}
We present the finalized formulation of the H₇ Coherence Geometry Law as a
proposed geometric law rather than a universal numerical constant. Across
diverse physical, biological, and synthetic systems, a common coherence band
repeatedly emerges when stability is evaluated through residual error after
canonical alignment, nonlinear weighting, and cusp-limited survival. We show
that this convergence reflects a scale-invariant fixed point of the induced
error geometry. The numerical value near $0.70$–$0.72$ is a
projection-dependent shadow of the geometry, not the law itself.
\end{abstract}

\section{README-Level Claim}

\textbf{Proposed Law.}  
H₇ is a geometric law describing a universal fixed point of coherence under
residual error dynamics.

\begin{itemize}
\item The law governs how stable systems organize under noise and drift.
\item It predicts convergence toward an interior coherence ridge.
\item The numerical band $\approx 0.70$–$0.72$ arises under a normalized
      projection and shifts predictably under reparameterization.
\item Breaking the geometry breaks the convergence.
\end{itemize}

\section{Context and Motivation}

Early Codex investigations identified a narrow coherence band recurring across
unrelated domains, including solar activity, turbulence, molecular lattices,
neural dynamics, and synthetic systems. This was initially interpreted as
evidence for a universal numerical constant.

Subsequent surrogate preservation tests and mapping variations demonstrated
that the convergence persists while the numerical value shifts. This motivated
a reformulation from a constant claim to a geometric fixed-point law.

\section{Canonical Residual Error Definition}

Let $\Delta\Phi(t)$ denote residual error after canonical alignment. Alignment
removes all arbitrary global offsets (phase, reference frame, baseline), such
that $\Delta\Phi$ represents irreducible local misalignment.

Normalized coherence is defined as:
\begin{equation}
C(t) = \frac{1}{1 + |\Delta\Phi(t)|}.
\end{equation}

This mapping is dimensionless, bounded, and monotonic.

\section{Nonlinear Weighting and Error Geometry}

Stable regions contribute disproportionately to long-term system behavior.
This asymmetry is encoded via nonlinear weighting:
\begin{equation}
\Omega(t) = \frac{1}{1 + |\Delta\Phi(t)|}.
\end{equation}

The weighting induces a curved error geometry in which coherent regions
dominate statistics and unstable regions are suppressed.

\section{Cusp-Limited Stability}

Empirical and simulated systems do not degrade smoothly beyond a threshold.
Instead, they exhibit cusp-like instability: once residual error exceeds a
critical region, collapse or escape rapidly follows.

This survival constraint limits the contribution of high-error states and is a
necessary component of the geometry.

\section{Toy Model Illustration}

To make the fixed-point mechanism explicit, consider a minimal model in which
$\Delta\Phi$ is drawn from a heavy-tailed distribution (e.g., lognormal or
Pareto), as commonly observed in critical systems. Applying nonlinear weighting
$\Omega$ and imposing a cusp-like survival cutoff yields an effective error
distribution whose expected coherence converges toward a stable interior
maximum.

Across wide parameter ranges, simulations demonstrate:
\begin{itemize}
\item Removing nonlinear weighting significantly lowers $\langle C \rangle$.
\item Alternative coherence mappings shift the fixed point predictably.
\item Heavier-tailed residuals push $\langle C \rangle$ lower unless the cusp
      constraint is strengthened.
\end{itemize}

These behaviors are characteristic of a geometric fixed point rather than a
numerical constant.

\section{Fixed-Point Emergence}

The combined effect of residual error measurement, nonlinear weighting, and
cusp-limited survival induces a flow in coherence space toward a stable ridge.

Across domains and simulations, the weighted surviving mean coherence
$\langle C \rangle$ converges toward:
\begin{equation}
\langle C \rangle \rightarrow C^\ast \approx 0.70\text{–}0.72.
\end{equation}

This value represents a fixed point of the geometry, not an imposed constant.

\section{Relation to Earlier H₇ Results}

Earlier empirical findings near $C \approx 0.70$ align with the analytic optimum
of:
\begin{equation}
U(C) = C^7(1-C)^3,
\end{equation}
which peaks at $C^\ast = 7/10$. This function is now understood as a specific
projection of the same stability geometry rather than an independent law.

\section{Representative Domain Examples}

\begin{center}
\begin{tabular}{lll}
\toprule
Domain & Observed $\langle C \rangle$ Band & Notes \\
\midrule
Solar X-ray flux & $\sim$0.72–0.745 & $\Omega$-weighted, cusp-limited intervals \\
Turbulence fields & $\sim$0.73 & Residual phase after alignment \\
Pink-noise extensions & $\sim$0.75–0.76 & Heavier tails, stronger cusp \\
Toy heavy-tail model & $\sim$0.68–0.69 & Matches analytic flow prediction \\
\bottomrule
\end{tabular}
\end{center}

\section{Universality Parallels}

The H₇ fixed point is analogous in spirit to other universality phenomena, such
as Feigenbaum constants in chaos or critical exponents in the Ising model. In
each case, microscopic details vary while macroscopic behavior converges under
a shared geometry.

\section{Falsification Protocol}

The proposed H₇ Coherence Geometry Law is falsifiable:

\begin{itemize}
\item Remove alignment $\rightarrow$ coherence disperses.
\item Remove nonlinear weighting $\rightarrow$ fixed point shifts or vanishes.
\item Remove cusp-limited survival $\rightarrow$ mean coherence decreases.
\item Alter the coherence mapping $\rightarrow$ numerical value shifts
      predictably.
\end{itemize}

Persistence of convergence under these violations would invalidate the law.

\section{Interpretation}

Nature does not optimize for maximal coherence. Perfect coherence is fragile.
Stable systems organize toward the highest coherence compatible with long-term
survival under noise and drift. That optimum is geometric, not absolute.

\section{Conclusion}

We propose the H₇ Coherence Geometry Law as a scale-invariant fixed-point
principle governing stable systems. The repeated appearance of
$\approx 0.70$–$0.72$ across domains reflects a universal geometric attractor
under residual-error dynamics, nonlinear weighting, and cusp-limited stability.
The law resides in the geometry; the number is its shadow.

\end{document}
